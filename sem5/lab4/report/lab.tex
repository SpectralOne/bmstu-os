\section*{Задание 1}

Процессы-сироты. В программе создаются не менее двух потомков. В потомках вызывается {\ttfamily sleep()}. Чтобы предок гарантированно завершился
раньше своих потомков. Продемонстрировать с помощью соответствующего вывода информацию об идентификаторах процессов и их группе.

\begin{lstinputlisting}[
	caption={Рекурсивный},
	style={go},
	]{../src/1.c}
\end{lstinputlisting}

\img{50mm}{1}{Демонстрация работы}

\section*{Задание 2}

Предок ждет завершения своих потомком, используя системный вызов
{\ttfamily wait()}. Вывод соответствующих сообщений на экран.

\begin{lstinputlisting}[
	caption={wait()},
	style={go},
	]{../src/2.c}
\end{lstinputlisting}

\img{50mm}{2}{Демонстрация работы}

\section*{Задание 3}

Потомки переходят на выполнение других программ. Предок ждет завершения своих потомков. Вывод соответствующих сообщений на экран.

\begin{lstinputlisting}[
	caption={execlp()},
	style={go},
	]{../src/3.c}
\end{lstinputlisting}

\img{80mm}{3}{Демонстрация работы}

\section*{Задание 4}

Предок и потомки обмениваются сообщениями через неименованный программный канал. Предок ждет завершения своих потомков. Вывод соответствующих сообщений на экран.

\begin{lstinputlisting}[
	caption={pipe},
	style={go},
	]{../src/4.c}
\end{lstinputlisting}

\clearpage
\img{80mm}{4}{Демонстрация работы}
\bigskip

\section*{Задание 5}

Предок и потомки обмениваются сообщениями через неименованный
программный канал. С помощью сигнала меняется ход выполнения программы. Предок ждет завершения своих потомков. Вывод соответствующих сообщений на экран.

\begin{lstinputlisting}[
	caption={Сигналы},
	style={go},
	]{../src/5.c}
\end{lstinputlisting}

\img{70mm}{5_1}{Демонстрация работы (без сигнала)}

\img{70mm}{5_2}{Демонстрация работы (с сигналом)}

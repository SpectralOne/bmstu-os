\chapter*{Задача <<Производство-потребление>>}


\section*{Листинги кода}


\begin{lstinputlisting}[caption={Очередь на основе циклического массива (буфера). Код.},label={lst:buffer},style={CStyle}]{../pc/src/buffer.c}

\end{lstinputlisting}

\begin{lstinputlisting}[caption={Очередь на основе циклического массива (буфера). Заголовочник.},label={lst:buffer},style={CStyle}]{../pc/include/buffer.h}

\end{lstinputlisting}

\begin{lstinputlisting}[caption={Реализация задачи. Код.},
label={lst:runners}, style={CStyle}]{../pc/src/runners.c}
\end{lstinputlisting}


\begin{lstinputlisting}[caption={Реализация задачи. Заголовочник.},
label={lst:runners}, style={CStyle}]{../pc/include/runners.h}

\end{lstinputlisting}


\begin{lstinputlisting}[caption={Точка входа в программу},
label={lst:main1},
style={CStyle}]{../pc/src/main.c}

\end{lstinputlisting}


\clearpage
\section*{Работа программы}


\img{165mm}{pc}{<<Производство-Потребление>>. Максимальная задержка потребителя -- 5с, производителя -- 2с.}



\chapter*{Задача <<Читатели-Писатели>>}


\section*{Листинги кода}


\begin{lstinputlisting}[caption={Реализация задачи. Код.},label={lst:io_obj},style={CStyle}]{../rw/src/io.c}

\end{lstinputlisting}


\begin{lstinputlisting}[caption={Реализация задачи. Заголовочник.},label={lst:constants2},style={CStyle}]{../rw/include/io.h}

\end{lstinputlisting}


\begin{lstinputlisting}[caption={Точка входа в программу},label={lst:main2},style={CStyle}]{../rw/src/main.c}

\end{lstinputlisting}

\clearpage
\section*{Работа программы}


\img{165mm}{rw}{<<Читатели-Писатели>>. Максимальная задержка -- 3с.}
